\chapter{Reductions to SAT}


\section{Introduction}
SAT is an NP-Complete problem. Therefore, all NP problems are reducible to SAT. This section covers the reductions of some well known hard problems to the SAT problem.

\section{Clique problem}
In graph theory, a clique in an undirected graph is a subset of its vertices such that every two vertices in the subset are connected by an edge. The clique problem which is solved here is the decision problem of testing whether a graph contains a clique larger than a given size.\\
\noindent \textbf{Problem}: Determine whether a graph $G$ on $n$ vertices has a clique of size $k$\\
\noindent \textbf{Solution} \cite{reduce}:\\
\noindent Convert the graph to a CNF using the following rules:\\
\noindent Variables: \\$y_{i,r}$ (true if node $i$ is the $r^{th}$ node of the clique) for $1 \leq i \leq n$, $1 \leq r \leq k$.\\
\noindent Clauses:
\begin{itemize}
\item For each $r$, $y_{1,r} \vee y_{2,r} \vee \ldots \vee y_{n,r}$ (some node is the $r^{th}$ node of the clique).

\item For each $i$, $r\leq s$, $\neg y_{i,r} \vee \neg y_{i,s}$ (no node is both the $r^{th}$ and the $s^{th}$ node of the clique).

\item For each $r \neq s$ and $i<j$ such that $(i,j)$ is not an edge of $G$, $\neg y_{i,r} \vee \neg y_{j,s}$. (If there's no edge from i to j then nodes i and j cannot both be in the clique).

\end{itemize}

\section{Hamiltonian path problem}

A Hamiltonian path is a path in an undirected or directed graph that visits each vertex exactly once.\\
\noindent \textbf{Problem}: Determine whether a graph $G$ on $n$ vertices has a Hamiltonian path\\
\noindent \textbf{Solution}:
\noindent Convert the graph to a CNF using the following rules:\\
\noindent Variables: \\$y_{i,r}$ (true if node $i$ is the $r^{th}$ node of the path) for $1 \leq i,r \leq n$.\\
\noindent Clauses:
\begin{itemize}
\item For each $r$, $y_{1,r} \vee y_{2,r} \vee \ldots \vee y_{n,r}$ (some node is the $r^{th}$ node of the path).

\item For each $i$, $r\leq s$, $\neg y_{i,r} \vee \neg y_{i,s}$ (no node is both the $r^{th}$ and the $s^{th}$ node of the path).

\item For each $i \neq j$ and $r<n$ such that $(i,j)$ is not an edge of $G$, $\neg y_{i,r} \vee \neg y_{j,r+1}$. (If there's no edge from i to j then they cannot be consecutive nodes in the path).

\end{itemize}

\noindent The above solution can be used to reduce the Hamiltonian cycle problem to SAT by replacing the clause $\neg y_{i,n-1} \vee \neg y_{j,n}$ by the clause $\neg y_{i,n-1} \vee \neg y_{j,1}$.

\section{Future work}

More interesting and hard problems like the Travelling Salesman Problem and the Discrete Logarithm Problem can be looked at.